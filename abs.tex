\begin{center}
    {\huge Caída de conos}\\
    {\normalsize David Amaro Alcalá}\\
    {\normalsize Laboratorio 2}\\
    dav1494@ciencias.unam.mx\\
\end{center}

\section*{Resumen}

Se dejarón caer tres conos con distinta masa pero la misma área
de contacto con el aire y se obtuvo su posición y el tiempo
con Tracker.

Se comprobo que la fuerza de resistencia del aire es directamente
proporcional al cuadrado y se obtuvo la constante de proporcionalidad.

Los valores obtenidos para la \textbf{velocidad terminal} son:

\begin{Tabla}
    \centering
    \begin{tabular}{|c|c|}
        \hline
        \rowcolor{azulito} Cono & Velocidad Terminal (m/s)  \\
        \hline 1 & -1.27384 \\
        \hline 2 & -1.42074 \\
        \hline 3 & -1.60727 \\
        \hline
    \end{tabular}
\end{Tabla}

Y también se obtuvieron los siguientes valores

\begin{Tabla}
    \centering
    \begin{tabular}{|c|c|}
        \hline 
        \rowcolor{azulito} Parámetro & Valor \\
        \hline $C_D$ & 0.776 \\
        \hline $r$ & 0.0095 \\
        \hline
    \end{tabular}
\end{Tabla}
