\section{Descripción}

\subsection{Material}

\begin{enumerate}
    \item Compás.
    \item Tijeras.
    \item Dos hojas de papel.
    \item Cámara digital (Resolución 10 \textbf{MP} a 30 \textbf{FPS}).
    \item Tripié.
    \item Metro (Resolución $1\times10^{-2}$ m).
    \item Plastilina.
    \item Balanza.
    \item Software \textit{Tracker}.
\end{enumerate}

\subsection{Montaje experimental}

\begin{description}
    \item[Paso 1] Se fabricaron los conos de tal forma que todos
        tuvieran un radio $a$ constante, lo demás era necesario variarlo.
    \item[Paso 2] Se midieron las masa de los dos conos utilizados.
    \item[Paso 3] Se coloco la cámara y una regla paralela al lente de
        la cámara.
    \item[Paso 4] Dejar caer el cono desde el reposo a una altura $h$ constante.
    \item[Paso 5] Análisis de los videos con Tracker.
\end{description}
