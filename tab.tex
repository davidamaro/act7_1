\section{Tablas}

\begin{table}
    \centering
    \begin{tabular}{|c|c|}
        \hline
        \rowcolor{azulito} Cono & Masa (kg) $\pm 5\times10^{-4}$ kg \\
        \hline Cono 1 & 0.00157 \\
        \hline Cono 2 & 0.00207 \\
        \hline Cono 3 & 0.00257 \\
        \hline
    \end{tabular}
    \caption{Masa de cada cono, medidas hechas con la
    báscula.}
    \label{tab:masa_cono}
\end{table}

\begin{longtable}{|c|c|c|c|}
        \hline
        \rowcolor{azulito} & Cono 1 & Cono 2 & Cono 3 \\
        \rowcolor{azulito} $t(s) \pm 0.0165$ s& $y(m) \pm 5\times10^{-4}$ m& $y(m) \pm  5\times10^{-4}$ m& $y(m) \pm 5\times10^{-4}$ m\\
        \endhead
        \hline 0 & 0 & 0 & 0 \\
        \hline 0.033 & -0.006 & -0.015 & -0.002 \\
        \hline 0.067 & -0.023 & -0.03 & -0.009 \\
        \hline 0.1 & -0.044 & -0.059 & -0.029 \\
        \hline 0.133 & -0.068 & -0.078 & -0.058 \\
        \hline 0.167 & -0.093 & -0.109 & -0.093 \\
        \hline 0.2 & -0.121 & -0.143 & -0.127 \\
        \hline 0.233 & -0.15 & -0.182 & -0.166 \\
        \hline 0.267 & -0.182 & -0.216 & -0.208 \\
        \hline 0.3 & -0.212 & -0.26 & -0.254 \\
        \hline 0.333 & -0.245 & -0.3 & -0.302 \\
        \hline 0.367 & -0.28 & -0.344 & -0.343 \\
        \hline 0.4 & -0.319 & -0.389 & -0.39 \\
        \hline 0.433 & -0.361 & -0.432 & -0.443 \\
        \hline 0.467 & -0.403 & -0.478 & -0.501 \\
        \hline 0.5 & -0.447 & -0.526 & -0.56 \\
        \hline 0.533 & -0.493 & -0.571 & -0.62 \\
        \hline 0.567 & -0.532 & -0.615 & -0.679 \\
        \hline 0.6 & -0.571 & -0.659 & -0.738 \\
        \hline 0.633 & -0.61 & -0.703 & -0.801 \\
        \hline 0.667 & -0.648 & -0.748 & -0.856 \\
        \hline 0.7 & -0.687 & -0.797 & -0.912 \\
        \hline 0.733 & -0.728 & -0.846 & -0.968 \\
        \hline 0.767 & -0.77 & -0.895 & -1.024 \\
        \hline 0.8 & -0.815 & -0.944 & -1.082 \\
        \hline 0.833 & -0.861 & -1.003 & -1.141 \\
        \hline 0.867 & -0.905 & -1.063 & -1.2 \\
        \hline 0.9 & -0.949 & -1.116 & -1.258 \\
        \hline 0.933 & -0.989 & -1.166 & -1.316 \\
        \hline 0.967 & -1.029 & -1.215 & -1.368 \\
        \hline 1 & -1.068 & -1.302 & -1.425 \\
        \hline 1.033 & -1.122 & -1.318 & -1.481 \\
        \hline 1.067 & -1.171 & -1.367 & -1.539 \\
        \hline 1.1 & -1.22 & -1.416 & -1.594 \\
        \hline 1.133 & -1.269 & -1.466 & -1.651 \\
        \hline 1.167 & -1.318 & -1.519 & -1.706 \\
        \hline 1.2 & -1.357 & -1.569 & -1.762 \\
        \hline 1.233 & -1.397 & -1.618 & -1.816 \\
        \hline 1.267 & -1.451 & -1.667 & -1.872 \\
        \hline 1.3 & -1.5 & -1.716 & -1.928 \\
        \hline 1.333 & -1.539 & -1.765 & -1.982 \\
        \hline 1.367 & -1.583 & -1.815 & -2.036 \\
        \hline 1.4 & -1.63 & -1.864 & -2.09 \\
        \hline 1.433 & -1.678 & -1.913 & -2.143 \\
        \hline 1.467 & -1.728 & -1.962 & -2.194 \\
        \hline 1.5 & -1.78 & -2.007 & -2.255 \\
        \hline 1.533 & -1.834 & -2.056 & -2.258 \\
        \hline 1.567 & -1.883 & -2.105 & -1.622 \\
        \hline
    \caption{Tabla con la posición y el tiempo de cada cono.}
    \label{tab:postiemcono}
\end{longtable}

Por la ecuación (\ref{int_caidalibre}) sabemos que la velocidad terminal es la
pendiente de la gráfica obtenida con los datos anteriores, entonces
se hace un ajuste lineal para obtener la ecuación de la recta y
su pendiente será el valor de la velocidad terminal. Las gráficas
pertenecen a la siguiente sección.

A continuación se listan las características de cada cono:

\begin{table}[h]
    \centering
    \begin{tabular}{c|c|c|}
        \cline{2-3}
        \rowcolor{azulito} & Altura (m) & Ángulo (rad) \\
        \hline \multicolumn{1}{|c|}{Cono 1} & 5 $\times 10^{-2}$ & $\frac{17 \pi}{45}$ \\
        \hline \multicolumn{1}{|c|}{Cono 2} & 3 $\times 10^{-2}$ & $\frac{14 \pi}{45}$ \\
        \hline \multicolumn{1}{|c|}{Cono 3} & 7 $\times 10^{-2}$ & $\frac{7 \pi}{18}$ \\
        \hline
    \end{tabular}
    \caption{Tabla con las carácteristicas que contaba cada cono
    al ser lanzado.}
    \label{tab:ConoCar}
\end{table}
